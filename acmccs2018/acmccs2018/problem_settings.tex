\section{Malware Information Retrieval System}

우리가 제안하는 malware IR 시스템은 벡터 학습 과 리트리벌 두 페이즈로 나뉜다. 두 페이즈에서 사용될 노테이션들을 정의하고, 이들을 이용해서 각 페이즈에서 어떤 태스크를 행해야하는지를 설명한다. 

\subsection{Notations}

우리는 먼저 학습 멀웨어 셋으로 Y를 갖고 있다. 그리고 각 멀웨어에 해당하는 싱글레이블 셋  Ys 과 멀티레이블 셋 Ym 이 있다. 이 레이블을 얻는 방법은 섹션 6.1에서 자세히 설명한다. 그리고 각 멀웨어로부터 손으로 추출된 피쳐의 튜플들의 셋을 V 로 정의한다. 어떤 피쳐들을 추출해서 사용했는지는 섹션 6.2에서 설명한다. 

멀웨어 IR 시스템은  4개의 원소를 갖는 tuple 로 정의된다. h 는 멀웨어 피쳐로부터 벡터 representation 을 얻을 수 있는 임베더 함수이다. E 는 검색이 될 멀웨어의 임베딩 벡터 셋이다. d 는 입력된 쿼리와 임베딩 벡터 간 거리를 계산하는 함수이다. R 은 리트리벌 결과를 반환하는 프레임워크이다. 학습 페이즈에서는 적절한 h 를 학습하고 이를 통해 Z로부터 E를 얻는다. 리트리벌 페이즈에서는 멀웨어 샘플 쿼리 q 를 입력받고 가장 가까운 k 개의 neighbor 를 랭킹 모듈 R 을 통해 반환한다.

% Table
\begin{table}%
\caption{Notations}
\label{tab:one}
\begin{minipage}{\columnwidth}
\begin{center}
\begin{tabular}{ll}
\toprule
Meaning & Notation\\
\midrule
  Set of Malwares     & $X$ \\
  ith malware sample  & $x_i$ \\
  Set of single labels & $Y_s$ \\
  Single label of ith malware    & $y_{si}$ \\
  Extracted features of ith malware & $v_i$ \\
  Set of extracted features   & $V$ \\
  Set of Multilabels   & $Y_m$ \\
  Multilabels of ith malware & $\vec{y}_{mi}$\\
  Neural embedder & $h$ \\
  Embedded malware vectors & $E$ \\
  Distance measuring function & $d$ \\
  Ranking Module & $R$\\
\bottomrule
\end{tabular}
\end{center}
\bigskip\centering
\footnotesize\emph{Source:} This is a table
 sourcenote. This is a table sourcenote. This is a table
 sourcenote.

 \emph{Note:} This is a table footnote.
\end{minipage}
\end{table}%



\subsection{Tasks}

학습페이즈에서는 IR system 에서 사용하기 위한 벡터 리프레젠테이션을 얻기 위한 Auxiliary Task를 수행한다. Auxiliary Task 는 멀웨어 분류 태스크를 학습하는 것으로 싱글 레이블 클래시피케이션이나 멀티레이블 클래시피케이션이 될 수 있다. f 인 f 는 멀웨어 피쳐로부터 레이블을 추정하는 함수이다. 함수 f 는 다시 g 와 h 의 합성 함수로 정의할 수 있다. h 는 위에서 설명한, 멀웨어 피쳐로부터 vector representation 을 얻을 수 있는 임베딩 함수이다. g는 임베딩으로부터 label 을 추정할 수 있는 클래시파이어이다.  
f(Z)

리트리벌 페이즈에서는 멀웨어 샘플 쿼리 q를 입력받아 학습 페이즈에서 구했던 h 함수를 이용하여 임베딩한다. 임베딩된 쿼리 eq 와 E 의 원소들 간 거리를 d 함수를 통해 측정하고, 가까운 k 개의 neighbor 를 랭킹 모듈 R 을 통해 반환한다. 
Result tuples = (Xvecj ) where


\subsection{Good Properties for Malware IR systems}
멀웨어 IR 시스템은 한계들이 존재했다. 특히 Raw binary file 로부터 좋은 피쳐를 뽑기가 수월하지 않고, intraclass variance 는 크고 innerclass variance 는 작으며 계속해서 변종과 새로운 종이 생겨나는 멀웨어 도메인의 특징으로 말미암아 좋은 멀웨어 IR 시스템이 가져야 할 속성들은 다른 도메인의 IR 시스템과는 조금 다르다.

\textbf{Semantic understanding. }
쿼링 샘플에 대해 구조적으로 비슷한 샘플 뿐만 아니라 의미적으로도 비슷한 샘플들을 랭킹하고 retrieve 할 수 있어야 한다. 여기서 의미가 비슷하는 것은 멀웨어의 행동 혹은 사람이 생각하기에 중요한 멀웨어의 속성들이 비슷하다는 것을 의미한다. 


\textbf{Robustness to novelty. }
새로운 변종들은 계속해서 나타나기 때문에 이에 빠르게 대응할 수 있어야 한다. 


\textbf{Robustness to rare inputs. }
멀웨어 도메인에서의 샘플 수는 그 멀웨어 패밀리의 영향력을 의미하지 않는다. 물론 비례하는 경향이 없는것은 아니지만, 하나의 멀웨어가 전세계적으로 유행할 수도 있고, 변종은 많지만 별로 영향력이 없는 멀웨어일 수도 있다. 따라서 해당 패밀리의 샘플 수가 적다고해서 쿼링 결과에서 랭킹 순위가 밀려서는 안된다. 즉 적은 수의 샘플에 대해서도 모델은 강건해야한다. 


\textbf{Robustness to polymorphism. }


\textbf{Robustness to variable size inputs. }
멀웨어의 raw file size Variance 는 매우 크다. 작게는 몇키로바이트부터 크게는 기가단위까지 갈 수 있다. 사이즈에 상관없이 retrieve 를 할 수 있어야 한다.

\textbf{Efficiency. }
세상에 존재하는 멀웨어 샘플 개수는 너무나도 많고 기하급수적으로 늘어나고 있기도 하다. 따라서 수많은 샘플들에서 k 개의 상위 랭크된 결과를 적절한 시간 내에 retrieve 하려면 랭킹 모듈이 효율적이어야한다.

