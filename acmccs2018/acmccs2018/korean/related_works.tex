\section{Related Works}

Nataraj et al. 은 large scale malware search and retrieval system을 제안하였다. 이 논문에서는 malware image로 부터 fingerprints를 얻고, 이를 nearest neighbor search를 통해 비슷한 샘플을 retrieval하는 방법을 제안하였다. 

Upchurch et al. 은 similarity testing을 통해 variant malware를 탐지하는 framework를 소개했다. 이 framework는 BitShred, TLSH, sdhash, ssdeep 등의 방법으로 정적 feature를 추출하고 이를 비교함으로써 유사한 malware인지의 여부를 판단하였다.

Palahan et al. 은 system call dependency graph로부터 significant malicious behaviors를 추출하고 이를 비교함으로써 malware간의 similarity를 비교하는 방법을 제시했다. 

Neural IR
document retrieval domain에서 많은 neural IR 모델들이 제안되었고 그 중 많은 모델들이 text의 좋은 representation을 얻기 위해 연구되었다[1),2),3)]. 

Multilabel embedding
Chih-Kuan et al. 은 multilabel embedding을 얻는 방법으로 label-correlation sensitive loss function을 사용하는 Canonical Correlated AutoEncoder 모델을 학습하였다. 논문에서는 dependency가 있는 label간의 관계를 End-to-End로 학습하고 이를 통해 Multi-label classification task를 해결하였고, 나아가 multilabel classification의 missing label 문제에 효과적임을 입증했다.
