\documentclass[sigconf, anonymous]{acmart}

\usepackage{kotex}% remove later
\usepackage{booktabs}% 써도됨?
\usepackage{mathtools}% 써도됨?

\fancyhf{} % Remove fancy page headers 
\fancyhead[C]{Anonymous submission \#9999 to ACM CCS 2017} % TODO: replace 9999 with your paper number
\fancyfoot[C]{\thepage}

\setcopyright{none} % No copyright notice required for submissions
\acmConference[Anonymous Submission to ACM CCS 2017]{ACM Conference on Computer and Communications Security}{Due 19 May 2017}{Dallas, Texas}
\acmYear{2017}

\settopmatter{printacmref=false, printccs=true, printfolios=true} % We want page numbers on submissions

%%\ccsPaper{9999} % TODO: replace with your paper number once obtained

\begin{document}
\title{Information Retrieval System for Malwares} % TODO: replace with your title

\begin{abstract}
In this paper, we describe the problems that arise when building a malware retrieval system using deep learning, and how to solve them. Multilabel Weighted Centerloss(MWC) is proposed to learn semantic distances between malware samples. Also we visualize the embedding vectors of malware samples and show the retrieval results to prove the synchronization between our perception of malwares and learned embedding space. \cite{medvinsky1993netcash}
\end{abstract}

% TODO: replace this section with code generated by the tool at https://dl.acm.org/ccs.cfm
\begin{CCSXML}
<ccs2012>
<concept>
<concept_id>10002978</concept_id>
<concept_desc>Security and privacy</concept_desc>
<concept_significance>500</concept_significance>
</concept>
<concept>
<concept_id>10002978.10002997.10002998</concept_id>
<concept_desc>Security and privacy~Malware and its mitigation</concept_desc>
<concept_significance>500</concept_significance>
</concept>
</ccs2012>
\end{CCSXML}

\ccsdesc[500]{Security and privacy~Malware and its mitigation}
% -- end of section to replace with generated code

\keywords{Malware, Information Retrieval, Neural Network} % TODO: replace with your keywords

\maketitle

\section{Introduction}


하루에 30만개 이상의 멀웨어가 생성되고 있고 이 숫자는 계속 늘어나고 있다. %ref
같은 기능을 하지만 해시는 다른 멀웨어들을 자동으로 제너레이션 하는 멀웨어 obfuscation 툴들의 발전 및 배포가 그 이유이다. 

반면 보안 전문가의 시간은 늘어나지 않고 있다. 보안 전문가의 업무 중 큰 부분을 차지하는 두 가지 일이 있다. 하나는 해시로 걸러지지 않은 악성코드의 카테고리를 분류하는 일이다. 두 번째는 주요 악성코드에 deep dive 해서 obfuscation 방법이나 악성 행위 방법을 공부하고 후에 나올 악성코드에 잘 대응할 수 있게 하는 것이다.   

이 두 가지 주요 업무 중에서 머신러닝이 쉽게 도움을 줄 수 있는 부분은 두 번째 태스크이다. 사실상 수많은 new 악성코드들을 보안 전문가들이 하나하나 정적 혹은 동적 분석을 통해 분류하는 일은 불가능하다. 따라서 자동으로 멀웨어를 분류하는 시스템은 예로부터 많이 만들었다. PE 스트럭쳐나 콜그래프 등의 정적 피쳐를 받아서 SVM, RF, DNN 등으로 멀웨어를 분류하는 모델이 연구되어왔다.   

하지만 이렇게 만들어진 분류기의 정확도가 만족스럽지 못하거나 새로운 샘플에 대해 대응을 잘 하지 못한다면 보안 전문가는 멀웨어 분류를 위해 학습된 머신러닝 모델을 사용하고나서 그 결과를 확인해야하는 수고를 해야한다. 그리고 그 확인 과정은 비슷한 멀웨어들을 검색해주는 시스템이 구축되어있지 않다면 수고스럽고 오래 걸린다. 

따라서 classification model 을 검증하는데에 사용할 수 있으면서도 동시에 위에서 언급한 보안 전문가의 두번째 태스크인 deep dive 에도 도움을 줄 수 있는 멀웨어 IR System 에 대한 니즈가 있다. 멀웨어 IR System 은 자연어 혹은 비젼의 IR System 과 마찬가지로 의미적으로 비슷한 멀웨어 샘플들을 랭킹하고, retrieve 해 줄 수 있는 시스템을 말한다. 따라서 우리는 멀웨어의 의미를 담을 수 있도록 멀웨어 IR 시스템을 구축해야한다. 

시스템이 Semantics - aware 라는 말은, 사람이 생각하는 멀웨어 샘플들간의 의미론적 관계가 시스템에 담겨있다는 뜻이다. 구조적으로 다르다고 해도 멀웨어의 행위가 일치하면 같은 의미를 갖는 샘플이라고 볼 수 있다. 마찬가지로 구조적으로 거의 같다고 해도 멀웨어의 행위가 전혀 다르다면 두 샘플은 다른 의미를 가졌다고 할 수 있다. 또한 샘플들 간 의미 차이 까지 고려될 수 있다면 그 시스템은 Semantics-aware한 시스템이라고 할 수 있다. 
 
Traditionally Semantic-aware Feature 를 사용해서 Semantics 를 담는 시스템을 구축하고자하는 시도가 있었다. 가장 멀웨어의 행위를 잘 나타낼 수 있는 동적 피쳐는 모든 경우의 수를 탐색하기 어렵기에 얻기가 어렵다는 단점이 있다. 정적 분석으로 얻을 수 있는 Syntatic 피쳐로 멀웨어 샘플의 의미를 표현하기 위해 보안 전문가가 피쳐 엔지니어링을 해서 특정 악성 행위의 공통적인 패턴을 조금 더 잘 검출하는 hand-crafted 피쳐를 획득할 수 있었지만, 복잡한 의미 관계까지 표현하는 머신러닝 모델을 만들기 위해서는 다양한 도메인에서 증명된 딥러닝같은 특징 표현 학습 기법을 멀웨어 도메인에 적용하는것을 고려해야한다.
 
딥러닝은 보안 전문가에 의해 피쳐 엔지니어링 된 피쳐들 뿐만 아니라, malware 바이너리, 소스코드, 메타데이터, 리소스, 동적 행위로그, 스크린샷 등 엔지니어링 되지 않은 인풋들에 대해서도 같은 의미를 갖는 멀웨어 샘플들을 표현하기 위한 공통 레이턴트 피쳐 표현을 하이러키컬하게 배울 수 있기에 Generalization 관점에서 이점이 있다. 또한 풍부한 Capacity 로 복잡한 의미관계를 학습할 수 있다는 장점도 있다. 

그렇다면 좋은 인풋 feature 와 딥러닝의 레이턴트 피쳐 표현 학습 능력이 있다면 멀웨어의 의미 관계까지 배울 수 있는가? 자연어 도메인에서는 의미 관계를 학습하기 위해서 문장에서의 앞 뒤 관계를 활용하거나[word2vec] 컨텍스트에 기반하여 co-occurence 를 확률적으로 표현한 메트릭스를 활용하는[GloVe?] 등, Self- supervised 방법을 활용한다. 하지만 멀웨어는 self-supervised learning 을 할 만한 요소가 없기 때문에, supervised learning 을 통해 의미 관계를 학습시키는 방법을 사용해볼 수 있다. 지금까지 알려진 대부분의 Deep Learning 기반 Malware Classifier는 어떤 파일이 어떤 malware family로 명명되어 있는 지를 분류하도록 가르친다. 하지만 이를 학습시켜서 얻은 representation latent feature 들은 거리가 가까우면 거의 동일한 멀웨어이고 거리가 멀면 동일하지 않은 멀웨어라는 것을 알 수 있을 뿐, 거리의 차이가 의미의 차이를 표현해주지는 않는다.  

그러면 어떻게 해야 멀웨어의 시멘틱을 딥러닝으로 학습할 수 있는가? 하나의 멀웨어가 가질 수 있는 악성 행위 혹은 속성은 여러개가 될 수 있다. 멀웨어의 시멘틱을 악성 행위 혹은 속성들이 의미하는 시멘틱 컴포넌트들의 조합이라고 생각해볼 수 있다. 따라서 우리는 멀웨어의 태그들을 정제하여 딥러닝 모델에게 제공하였고 이를 통해 딥러닝 모델이 학습할 멀웨어의 Semantics를 지도해주었다. 학습 과정에서는 멀웨어의 표현 벡터를 시멘틱 컴퍼넌트 벡터들의 리니어 컴비내이션이 되도록 학습하여 특정 의미를 갖게 하였다. 또한 우리가 제안한 메트릭 러닝 방법을 통해, 벡터 스페이스에서의 멀웨어 샘플들 간 거리가 의미의 차이와 같아지도록 학습하였다. 그리하여 멀웨어 샘플의 의미(Semantics)를 벡터로 표현할 수 있는 멀웨어 시멘틱 스페이스를 얻을 수 있었다. 

우리가 제안한 메트릭 러닝 방법은 센터로스[cite]를 어그멘트해서 설계한 Multilabel Centerloss Obejctive function을 크로스 엔트로피 로스와 함께 사용하는 것이다. 이 로스 펑션의 역할은 멀웨어의 Representation vector 가 그것의 시멘틱스 벡터와 가깝게 하는 것이다. 이 것의 장점은 어디에든 쉽게 붙일 수 있다는 것이며 구현이 어렵지 않다는 것이다. 더불어 Information Retrieval Task 에서 사용하던 Scoring 기법과 동이랗게, 특정 시멘틱 컴포넌트에 Weight를 부여하여 다른 컴포넌트들보다 랭킹에 큰 역할을 할 수 있도록 제약을 줄 수 있다.  

이렇게 학습된 Semantic Space에서 의미적으로 유사한 멀웨어 또는 어떤 semantic의 차이를 갖는 malware, 혹은 특정 의미를 갖는 멀웨어들을 검색할 수 있는 유연한 IR system을 소개한다. 더불어 딥러닝을 활용하여  그리고 정량 평가와 정성 평가를 통해 시멘틱 스페이스가 얼마나 잘 만들어졌는지 확인한다.  

우리의 메인 컨트리부션은 이렇게 요약할 수 있다.

\begin{itemize}
\item 우리는 악성코드의 Semantics를 벡터로 모델링하고 어떤 악성코드의 latent feature representation을 semantic components의 linear combination으로 표현할 수 있는 Semantics Space 위에 임베딩했다. 그리고 이를 학습하기 위해 Deep Learning 기반 Malware Classifier를 사용한다.
\item 우리는 Malware Classifier가 학습하는 Semantic Space에서 metric function이 semantics의 차이를 의미하도록 학습하기 위해 multilabel centerloss에 기반한 Metric Learning 기법을 제안한다.
\item 학습된 Semantic Space에 존재하는 멀웨어 샘플 벡터들에 대해 멀웨어, 시멘틱, 멀웨어와 시멘틱의 조합 세 가지 방법으로 malware를 검색할 수 있는 유연한 IR system을 소개하고 그 성능을 평가한다.
\end{itemize}

페이퍼 구성은 다음과 같다. 딥러닝, IR, Metric learning, Semantic Space 등 주요 배경 지식에 대한 설명하는 Background 가 섹션 2에 위치한다. 섹션 3는 멀웨어 IR 시스템에 대한 구조와 desired properties 를 정의한다. 섹션 4에서는 우리가 학습할 멀웨어 시멘틱 스페이스를 설명하고 이를 학습하기 위한 우리의 제안을 보여준다. 섹션 5에서는 실험에 쓰인 데이터셋과 뉴럴넷 하이퍼파라미터 설정들, 그리고 정량 평가와 정성 평가 실험 결과에 대해서 설명한다. 섹션 6에서는 related works 를, 섹션 7에서는 Future works 그리고 세션 8 에서는 conclusion 이 위치한다. 

% contribution 에 centerloss 어그멘트했다고쓰기.

\section{Background}

% Handcrafted Feature Extract
\subsection{Deep Learning }

Neural networks are machine learning implementations that extract features based on statistical characteristics of training data. DNNs also provide hierarchical feature representations by configuring the network into several hidden layers. Generally, a DNN consists of an input layer, several hidden layers, and an output layer. For a classification task, the input layer receives a feature vector representation of the target sample of the task as input. The output layer outputs the probability of a class related to a given input vector as a vector. This allows the DNN to predict the class per the input vector.
 
The image classification task can improve the performance of the model by using the convolutional neural network (CNN) \cite{krizhevsky2012imagenet}, consisting of convolutional layers, pooling layers, and general neural network layers. For the convolutional layer, we extract features by applying a convolutional filter to the input. The pooling layer has the effect of down-sampling the features giving less effect.


\subsection{Information Retrieval}
Information retrieval (IR) was introduced in natural language processing as a task to find related documents from a string query. However, it has been extended to other domains and is now used in various fields (e.g., image \cite{datta2008image, yu2015learning}, music \cite{schedl2014music}, medical care \cite{goeuriot2016medical, mourao2015multimodal}, and malware \cite{santos2013noa}).

\textbf{Boolean model.} A Boolean model is a classical information retrieval model. It is a search model that considers documents and user queries as set from the background of boolean logic and set theory. Whereas its ease-of-use and intuitive implementation is advantageous, it is limited because the similarity is calculated discretely, and the retrieval result becomes very large or cannot be retrieved if it does not exactly match. Furthermore, all terms have the same weight \cite{lashkari2009boolean} .
	


\textbf{Vector space model.} Another method is the vector space model. This model represents documents and user queries in a vector space so that the distance between vectors can become similar. This requires a simple linear algebra model. It improves the retrieval result by giving different weights to the terms; it shows the similarity of the query as a continuous value. There are various models according to how to make the query of the document and the user into the vectors \cite{guo2016deep, liu2009learning} and how to measure the distance between the vectors \cite{diaz2016query, huang2013learning, roy2016using, mitra2016dual}.


\textbf{Learn vector representations using deep learning.} As mentioned, using deep learning to perform machine learning tasks results in better generalization. This is because it learns via hierarchical feature representations. Therefore, the combination of various tasks using the vector space model and deep learning results in performance breakthroughs for many tasks \cite{krizhevsky2012imagenet, mikolov2013efficient, hinton2012deep}. Additionally, the IR field using the vector space model also achieves a breakthrough with deep learning \cite{huang2013learning, severyn2015learning, wan2014deep}.


There are supervised and unsupervised methods for learning vector representations. A typical unsupervised learning method uses an autoencoder. Attempts have been made to encode malware using the autoencoder. Supervised learning occurs when a person provides an answer label, and a classification or regression model performs the task of matching the label. The output of the bottleneck layer obtained from the learning process as a representation vector of the corresponding sample contains the most compressed information. This allows some control over the position of the vector representation. For example, one can use the objective function for metric learning to determine which samples to locate near and which samples to locate far. This allows one-shot or few-shot learning tasks, improving the performance of information retrieval and determining rankings based on the distance between vectors and respond to fewer samples.


\subsection{Metric Learning}
Metric learning is a technique for learning a distance function. More specifically, through metric learning, one can train a distance function to determine which samples will be close to each other, which samples will be farther, or, to some extent, how far the distance will be.

A typical example of metric learning using deep learning is the Siamese network. The Siamese network receives pairs of samples as input data, teaching that the distance of the feature vectors is 0 if the pairs are the same label, and that the distance is 1 or more if the labels are different.

Another example is centerloss \cite{wen2016discriminative}. This creates a template vector for each class in the single-label classification task and adds constraints to the existing cross-entropy loss so that the representation vectors of the samples belonging to the same class become the same as the template vector of the corresponding class. This reduces the intra-class variance so that representation vectors can be used for tasks like information retrieval.

The distance function learned by metric learning is used in many real-word tasks to find similar data, such as face recognition, image retrieval, and text retrieval \cite{sun2014deep, huang2013learning, wan2014deep}.


\subsection{Semantic Spaces for Natural Language}

The purpose of using semantic spaces is to represent a natural language that captures meaning in the natural language domain. The original motivation of using semantic spaces was to address two challenges in the natural language domain. The first is to make words with similar syntactics but different meanings to different representations, and the other is to make words with the same meaning but the different syntactics to the same representation.

Recently, neural networks have made major advances in combination with other new approaches (e.g., word2vec \cite{mikolov2013efficient},  GloVe \cite{pennington2014glove} and FastText \cite{joulin2016fasttext}). If we can express the natural language well in semantic spaces, we can use it for various tasks, such as document classification, document search, question and answering, speech recognition, and translation.


\section{Malware Information Retrieval System}
% 목표하는 바

우리가 제안하는 malware IR 시스템은 벡터 학습 과 리트리벌 두 페이즈로 나뉜다. 두 페이즈에서 사용될 노테이션들을 정의하고, 이들을 이용해서 각 페이즈에서 어떤 태스크를 행해야하는지를 설명한다. 


\subsection{Notations}

우리는 먼저 학습 멀웨어 셋으로 $X = \{x_1, x_2, ..., x_N\}$ 를 갖고있다. 그리고 각 멀웨어에 해당하는 싱글레이블 셋 $Y_s = \{y_{s1}, y_{s2}, ..., y_{sN}\}$ 과 멀티레이블 셋 $Y_m = \{\vec{y}_{m1}, \vec{y}_{m2}, ... , \vec{y}_{mN}\}$ 이 있다. 이 레이블을 얻는 방법은 섹션 6.1에서 자세히 설명한다. 그리고 각 멀웨어로부터 손으로 추출된 피쳐들의 셋을 $V = \{v_1, v_2, ..., v_n \}$ 로 정의한다. 어떤 피쳐들을 추출해서 사용했는지는 섹션 6.2에서 설명한다. 

멀웨어 IR 시스템은 $[h, E, d, R]$ 4개의 원소를 갖는 tuple 로 정의된다. $h$ 는 멀웨어 피쳐로부터 벡터 representation 을 얻을 수 있는 임베더 함수이다. $E$ 는 검색이 될 멀웨어의 임베딩 벡터 셋이다. $d$ 는 입력된 쿼리와 임베딩 벡터 간 거리를 계산하는 함수이다. $R$ 은 리트리벌 결과를 반환하는 프레임워크이다. 학습 페이즈에서는 적절한 $h$ 를 학습하고 이를 통해 $Z$로부터 $E$를 얻는다. 리트리벌 페이즈에서는 멀웨어 샘플 쿼리 $q$ 를 입력받고 가장 가까운 $k$ 개의 neighbor 를 랭킹 모듈 $R$ 을 통해 반환한다.

% Table
\begin{table}%
\caption{Notations}
\label{tab:one}
\begin{minipage}{\columnwidth}
\begin{center}
\begin{tabular}{ll}
\toprule
Meaning & Notation\\
\midrule
  Set of Malwares     & $X$ \\
  ith malware sample  & $x_i$ \\
  Set of single labels & $Y_s$ \\
  Single label of ith malware    & $y_{si}$ \\
  Extracted features of ith malware & $v_i$ \\
  Set of extracted features   & $V$ \\
  Set of Multilabels   & $Y_m$ \\
  Multilabels of ith malware & $\vec{y}_{mi}$\\
  Neural embedder & $h$ \\
  Embedded malware vectors & $E$ \\
  Distance measuring function & $d$ \\
  Ranking Module & $R$\\
\bottomrule
\end{tabular}
\end{center}
\bigskip\centering
\footnotesize\emph{Source:} This is a table
 sourcenote. This is a table sourcenote. This is a table
 sourcenote.

 \emph{Note:} This is a table footnote.
\end{minipage}
\end{table}%



\subsection{Tasks}

학습페이즈에서는 IR system 에서 사용하기 위한 벡터 리프레젠테이션을 얻기 위한 Auxiliary Task를 수행한다. Auxiliary Task 는 멀웨어 분류 태스크를 학습하는 것으로 싱글 레이블 클래시피케이션이나 멀티레이블 클래시피케이션이 될 수 있다. $f: V \rightarrow Y $ 인 f 는 멀웨어 피쳐로부터 레이블을 추정하는 함수이다. 함수 $f$ 는 다시 $g$ 와 $h$ 의 합성 함수로 정의할 수 있다. $h$ 는 위에서 설명한, 멀웨어 피쳐로부터 vector representation 을 얻을 수 있는 임베딩 함수이다. $g$는 임베딩으로부터 label 을 추정할 수 있는 클래시파이어이다.  
\[
f(v_i) = (g \circ h)(v_i) = g(h(v_i)) = g(e_i) = y_i 
\]
where
\[
g(E) = \sigma (w*e_i + b) 
\]

리트리벌 페이즈에서는 멀웨어 샘플 쿼리 q를 입력받아 학습 페이즈에서 구했던 h 함수를 이용하여 임베딩한다. 임베딩된 쿼리 eq 와 E 의 원소들 간 거리를 d 함수를 통해 측정하고, 가까운 k 개의 neighbor 를 랭킹 모듈 R 을 통해 반환한다. 

% Algorithm 으로 대체
\[
Results = \{x_j | j = argmin_i( {-d(e_q, e_i)} )  \}
\]

\subsection{Desired Properties for Malware IR systems}
멀웨어 IR 시스템은 한계들이 존재했다. 특히 Raw binary file 로부터 좋은 피쳐를 뽑기가 수월하지 않고, intraclass variance 는 크고 innerclass variance 는 작으며 계속해서 변종과 새로운 종이 생겨나는 멀웨어 도메인의 특징으로 말미암아 좋은 멀웨어 IR 시스템이 가져야 할 속성들은 다른 도메인의 IR 시스템과는 조금 다르다.

\textbf{Semantic understanding. }
쿼링 샘플에 대해 구조적으로 비슷한 샘플 뿐만 아니라 의미적으로도 비슷한 샘플들을 랭킹하고 retrieve 할 수 있어야 한다. 여기서 의미가 비슷하는 것은 멀웨어의 행동 혹은 사람이 생각하기에 중요한 멀웨어의 속성들이 비슷하다는 것을 의미한다. 


\textbf{Robustness to novelty. }
새로운 변종들은 계속해서 나타나기 때문에 이에 빠르게 대응할 수 있어야 한다. 


\textbf{Robustness to rare inputs. }
멀웨어 도메인에서의 샘플 수는 그 멀웨어 패밀리의 영향력을 의미하지 않는다. 물론 비례하는 경향이 없는것은 아니지만, 하나의 멀웨어가 전세계적으로 유행할 수도 있고, 변종은 많지만 별로 영향력이 없는 멀웨어일 수도 있다. 따라서 해당 패밀리의 샘플 수가 적다고해서 쿼링 결과에서 랭킹 순위가 밀려서는 안된다. 즉 적은 수의 샘플에 대해서도 모델은 강건해야한다. 


\textbf{Robustness to polymorphism. }


\textbf{Robustness to variable size inputs. }
멀웨어의 raw file size Variance 는 매우 크다. 작게는 몇키로바이트부터 크게는 기가단위까지 갈 수 있다. 사이즈에 상관없이 retrieve 를 할 수 있어야 한다.

\textbf{Efficiency. }
세상에 존재하는 멀웨어 샘플 개수는 너무나도 많고 기하급수적으로 늘어나고 있기도 하다. 따라서 수많은 샘플들에서 k 개의 상위 랭크된 결과를 적절한 시간 내에 retrieve 하려면 랭킹 모듈이 효율적이어야한다.


\begin{figure*}[!htb] % concept
  \includegraphics[width=\textwidth]{../../figures/concept.pdf}
  \caption{concept}
  \label{fig:concept}
\end{figure*}

\section{Semantics Aware Representation Learning}
이번 섹션에서는 Auxiliary task 를 통해 malware를 멀웨어 시멘틱 스페이스 위에서 해석할 수 있게 임베딩시키는 방법을 설명한다. 먼저 우리가 정의한 멀웨어 시멘틱 스페이스가 무엇인지 설명하고, semantic-aware malware representation vector 를 학습하기 위한 Loss function 을 제안한다.

\subsection{Semantic Spaces for Malwares}

\textbf{Definition. }
Let $X = \{x_1, …, x_n, …, x_N\}$ be a set of malwares.
Let $S$ be a subset of vector space $V$ of dimension $d$ and let semantic component set $S = \{s_1, ... , s_k, … s_K\}$ be a linearly independent subset of $V$.  
Then for every $\mathbf{e} \in E$, there is an unique linear combination of the sementic component vectors that equals $e$.
\[
\mathbf{e} = c_1\mathbf{s_1} + c_2\mathbf{s_2} + … + c_k\mathbf{s_k} 
\]

where $c_i$ is the i’th element of coefficient vector $ \mathbf{c} = \{c_1, ... , c_k\}$
We call a vector $e$ as a semantic vector of malware $x$ and there is a nonlinear mapping from $X$ to $E$. 

\textbf{Metric function. }
we use an Euclidean Distance $d(e_i, e_j)$ for a metric of a set E and it means the function that defines a distance between  semantics of two malwares. 

\textbf{The meaning of a learning malware semantic space. }
Malware semantic space 를 배운다는 것은 무슨 뜻일까.
언어를 예를 들어보자. 우리는 머릿속으로 어떤 단어들의 의미을 생각해볼 수 있다. 그 의미들의 관계를 생각해보자. 도시 이름들과 머신 러닝에 관련된 단어들은 멀리 위치한다고 생각해볼 수 있다. 그리고 각각은 가까울 것이라고 생각할 수 있다. 단어들 간 관계의 유사성을 생각해볼 수 도 있다. King 과 Queen 의 관계는 Man 과 Woman 의 관계와 비슷하다는 생각을 해볼 수 있다. 이렇게 머릿 속에 존재하는 단어들의 의미를 최대한 approximate 해서 표현하려면, 벡터 스페이스 위에 개념들을 벡터로서 올려보는 방법이 있다. 의미가 가깝다면 벡터 간 거리가 가깝게,  의미가 멀다면 벡터 간 거리가 멀게 위치시킬 수 있다. king - queen = man - woman 와 같이, 벡터스페이스에서 정의된 연산을 이용해서 우리가 머릿속으로 생각하는 단어들 간의 관계를 모사(approximate)하여 표현해볼 수도 있다. 

마찬가지로 우리는 머릿속에서 악성코드의 의미를 생각해볼 수 있다. 우리는 악성코드가 악성 컴포넌트들의 조합이라고 여기고, 이를 벡터 스페이스의 연산으로 모사(approximate)하여 표현하기 위해, basis 와 Linear combination 이라는 개념을 이용하였다. 즉, 멀웨어의 sematnic vector 가 semantic component vector의 선형 조합으로 표현되도록 가르치는 것은 우리의 머릿 속에 존재하는 Semantics 를 벡터 스페이스에서 최대한 따라하겠다는 것이고 이것의 의미가 바로 Semantic space 를 학습한다는 의미이다. 

\textbf{Evidences. } 우리가 위에서 정의한대로 Semantic space를 잘 학습했다면 다음 쿼링 태스크들이 제대로 작동해야한다. 먼저 멀웨어 샘플로 쿼링해서 의미가 비슷한 샘플들을 검색할 수 있어야 한다. 둘째로, semantic components의 조합 만으로 검색할 수 있어야 한다. 이를테면, ransom, downloader, agent 세 가지 속성을 모두 갖는 멀웨어를 검색할 수 있어야한다. 마지막으로, 샘플과 semantic components 의 조합으로 검색할 수 있어야 한다. 이를테면, ransom, downloader 의 속성을 갖는 샘플과 agent 라는 semantic component 를 함께 검색하여 세 속성 모두 갖는 샘플들을 검색할 수 있어야 한다. 이 쿼리 방법들에 대한 그림 설명은 Figure.\ref{fig:qualitative_all} 에서 확인할 수 있다. 

\subsection{Solution Overview}
\begin{figure*}[!htb] % qualitative_all
  \includegraphics[width=\textwidth]{../../figures/qualitative_all_fix.pdf}
  \caption{Queryings}
  \label{fig:qualitative_all}
\end{figure*}

\textbf{Multilabel Classification. }
멀웨어 샘플의 의미를 반영하는 Vector representation E 를 학습시키기 위해 멀티레이블 분류 학습을 Auxiliary Task 로 선택하였다. 이 분류 태스크를 학습하기 위해 뉴럴 임베더 h는 표현 벡터들을 선형 분류기가 분류할 수 있도록 위치시킨다. 기존 대부분의 멀웨어 분류기를 학습하는 연구들은 하나의 레이블로 분류하도록 분류기를 학습한다. 

하지만 멀웨어 도메인에서 하나의 레이블을 특정하고 이를 기준으로 가르치는 Auxiliary Task로 Malware IR System 을 만든다면, 학습하는 임베딩 벡터가 그 샘플의 의미를 담기에 부족한 점이 있다. 첫 째, 멀웨어에 레이블링을 하는 프로세스가 멀웨어의 의미를 표현하는데에 합리적이지 못하다. 멀웨어는 여러 악성 행위들을 동시에 하는 경우가 많다. 하지만 멀웨어를 대분류, 중분류, 소분류 등 하이러키컬한 분류 체계를 갖도록 레이블링 하는 경우가 대부분이다. 또한 네이밍 규칙이 글로벌하게 일관되지 못하고 심지어 로컬하게도 분석가들 마다 다른 기준으로 네이밍을 하는 경우가 많다. 유행하는 멀웨어에 대해서는 그 멀웨의 레이블을 붙일 때 그 멀웨어의 행위와 상관 없는 특징을 이름으로 붙이기도 한다. 이러한 레이블링 프로세스들은 머신러닝 모델이 멀웨어를 입력으로 받아서 레이블을 예측하는 Supervised Learning Task 를 수행할 때 멀웨어의 의미를 배우는 것을 방해한다. 

둘 째는 하나의 레이블이 만약 Trojan이나 Adware 같은 대분류라면 같은 분류 안의 샘플들을 좀 더 세세하게 구분하여 information retrieval task 를 수행할 때에 정확하게 같은 의미의 샘플을 검색하기가 쉽지 않다. 샘플들의 벡터의 차이가 정확하게 어떤 의미인지를 알 수 없기 때문이다. 레이블이 만약 소분류라면 거의 동일한 샘플들만 같은 분류 안에 속해있을 가능성이 높고, 분류의 개수가 너무 많기 딥러닝 모델의 캐퍼시티가 충분하다면 샘플들을 외우고 하이러키컬 representation 을 학습하지 않아 Generalization 효과가 떨어지게 될 것이다. 따라서 정확하게 같은 샘플들 혹은 아주 약한 변종들이 검색될 수 있지만 의미가 비슷한, 동일하지 않은 샘플들에 대해서는 검색되지 않을 가능성이 높다. 

우리는 같은 의미를 가진 스트럭쳐가 다른 샘플들 뿐만 아니라 비슷한 의미를 가진 멀웨어 샘플들에 대해서도 Retrieval이 가능하도록 하기 위해, 멀티 레이블을 분류하도록 Auxiliary Task 를 학습시키고, 거기에서 얻은 Vector representation 을 검색 시스템에서 사용한다. 이는 여러 의미 계층에 대한 공통 피쳐를 딥러닝 모델이 모두 학습할 수 있도록 도와준다. 


\textbf{Centerloss. }
위의 방식대로 Multilabel Classification 을 학습시키면서 임베딩한 멀웨어의 Representation vector 들이 우리가 정의한 Semantic Space 위에 존재하도록 하려면 SingleLabel 센터로스를 발전시켜 만든 Multilabel Centerloss(MCL) 를 사용해야 한다. 
The first reason is about the variance of samples in the same class. 일단 기존 센터로스의 효과는 인트라 클래스의 베리언스를 낮춰주는 것이라고 백그라운드에서 설명했다. 약간 더 부연 설명을 하자면, 센터로스를 사용하면 inner class variance 가 너무 커서 ranking module 에서 같은 클래스 내 두 샘플의 거리가 서로 다른 클래스 내의 두 샘플 간 거리보다 더 커지는 현상을 막아주어 결과적으로 IR 결과를 향상시킨다. 

두 번째 이유는 우리가 정의한 시멘틱 스페이스의 성질을 만족시키도록 멀웨어 representation vector 를 임베딩 시킬 수 있다는 것이다. 싱글레이블 센터로스를 제안했던 논문에서는 각 레이블 별로 템플릿 벡터를 익스터널 메모리에 저장해놓고 그 템플릿 벡터와 보틀넥의 거리가 작아지도록 MSE 에러를 설정한다. 이는 곧 템플릿 벡터의 속성을 정의하지는 않았지만, 특정 클래스에 해당하는 샘플의 보틀넥은 그 클래스에 해당하는 템플릿 벡터가 되도록 가르칠 수 있다는 것을 의미하고 이 부분이 우리가 센터로스를 사용하여 우리의 문제를 해결해야겠다고 영감받은 부분이다. 마찬가지로 각 시멘틱 컴포넌트 별로 벡터를 익스터널 메모리에 저장해놓고, 보틀넥이 시멘틱 컴포넌트 벡터들의 linear combination 이 되도록 MSE 에러를 로스펑션으로 설정했다. 이 로스에 의해서 우리는 멀웨어의 시멘틱스 벡터가 시멘틱 컴포넌트의 리니어 컴비내이션으로 만들어지는 것을 학습시킬 수 있다. 

%그리고, 보틀넥과 시멘틱 컴포넌트들의 합의 차이에 alpha/the number of answer labels 만큼의 곱을 하여 각 레이블을 업데이트해준다. 알파는 하이퍼파라미터로, 템플릿 벡터와 보틀넥을 얼마나 섞어서 기존 템플릿벡터를 업데이트할지를 결정해주는 값이다. 

\textbf{Learn to Rank. }
멀웨어의 레이블 중에는 우리가 중요하게 여기는 레이블들과 아닌 레이블들이 있다. 멀웨어 retrieval 시스템을 만들 때 이 중요도를 고려하는 것은 semantic understanding property 를 만족시키기 위한 중요한 요소 중 하나이다. 예를 들어 PE 포멧의 레이블들 중, agent 나 downloader보다 ransom, coinminer 등의 레이블에 더 가중치를 주어서 랭킹을 한다면 더 semantics-aware 한 검색 시스템이라고 여길 수 있을 것이다. 이렇게 어떤 시멘틱 컴포넌트에 대해서 더 잘 검색되게 할 지를 IR 시스템을 만들 때 반영하기 위해, 우리는 태그들의 중요도를 constraint 로 추가한 Centerloss 를 사용한다. 여기에서 Constraint 는 시멘틱 컴포넌트들의 Norm을 결정짓는다. 우리가 리트리벌에서 사용하는 Metric function 인 유클리디언 디스턴스는 거리를 계산할 때 두 벡터의 Norm 에 영향을 받는다. Norm 이 큰 semantic component vector 와 다른 컴포넌트 벡터와의 거리는 Norm 크지 않은 컴포넌트 벡터들간의 거리보다 더 크게 될 가능성이 높다. 따라서 euclidean distance 로 랭킹을 하는 IR 시스템에서는 중요도가 큰 semantic component 를 갖고있는 멀웨어를 검색했을 때, 해당 semantic component 를 포함하는 멀웨어 샘플들이 검색될 가능성이 더 높아지게 된다. 이는 기존 IR 시스템에서 특정 속성을 가진 문서들이 더 잘 검색되게 만드는 Scoring 기법 (ex. tf-idf) 과 같은 역할을 하는 것이라고 볼 수 있다. 

%원래 리트리벌에서 랭킹을 위한 스코어링은 원래 하는짓인데 이걸 딥러닝에서 되게 한건 노블한거다. 연역적으로 왜 이렇게 해야하는지는 자명하다.


%\textbf{Semantic space }
%뭘 검색할 수 있어야 우리가 앞에서 문제라고 했던 애들이 해결된다. 연산한걸 검색할 수 있고,...




\subsection{Proposed Objective Functions}
Overview 에서 설명했던 방법들로 말미암아 시멘틱 스페이스를 배우기 위한 새로운 objective function 을 제안한다.  

\textbf{Multilabel Center Loss(MCL). }
우리가 풀고자 하는 constained optimization problem 의 식은 Eq.\ref{eqn:optimization}와 같다. 이는 cross-entropy loss의 근간이 되는 Negative Log Likelihood function 최적화 문제에 target semantic vector 와 representation vector 의 거리가 $\epsilon$ 미만이어야 한다는 constraint 를 추가한 것이다. 

\begin{equation}
\label{eqn:optimization}
\min_{\theta, w, b} J(\theta, w, b) = -\sum_i{ \sum_j{ y_{mij} \log{\hat{y_{ij}}}}}
\end{equation}

s.t.
\[
h(v_i;\theta)) - \mathbf{s}_\text{target} < \epsilon ,
\]
%\[
%||c_i||_2 = cc_i
%\]
for $i \in \{1,2, ..., N\}$ where $\epsilon >= 0$ and $s_\text{target}$ is a target semantic vetcor.

위의 식을 만족시키는 parameters 를 찾기 위해 cross-entropy loss 와 multilabel centerloss 를 1: lambda 의 비율로 더해서 최종 loss function 을 만들었다\cite{}. 우리가 제안하는 loss function 을 수식으로는 Eq.\ref{eqn:centerloss}과 같이 formulate할 수 있다. 

\begin{equation}
\label{eqn:centerloss}
\begin{aligned}
L &= L_s + \lambda L_c \\
 &= -\frac{1}{N}\sum_i{\sum_j{ y_{mij} \log{\hat{y_{ij}}}}} 
+ \lambda \frac{1}{N} \sum_i{( \mathbf{s}_{\text{target}} - h(v_i;\theta))^2}\\
\end{aligned}
\end{equation}

where 
\[
\hat{y_{ij}} = \frac{exp(Wh(v_i;\theta)+\mathbf{b})}{ 1 + exp(Wh(v_i;\theta)+\mathbf{b})}
\]

%$L_c = \sum_i^N{||e_i - s_\text{target}||^2}$

  
\textbf{Weighted Multilabel Center Loss(WMCL). }
앞서 설명한 것 처럼 우리의 Malware Information Retrieval System 에 Learn to rank 기법을 적용시키기 위해 Importance coefficients 를 각 시멘틱 컴포넌트 별로 부여한다. 기존 Multilabel Center Loss 에 semantic component vector 의 norm 과 importance coefficient 값이 같도록 하는 constraint 를 하나 더 추가하여 중요도를 반영하였다.
\[
||s_i|| = c_i \text{ for all i }\in \{i = 1,2, ..., M\}
\]


cross-entropy 로스는 multilabel classification 의 accuracy 를 높이기 위해 $\theta, W, \mathbf{b}$의 파라미터들을 업데이트시킨다. multilabel centerloss 을 통해 업데이트하는 대상은 크게 두 가지가 있다. 하나는 embedding vector 가 $\mathbf{s}_{\text{target}}$ 와 가까워지도록 $\theta, W, \mathbf{b}$ 의 parameter들을 back propagation 을 통해 업데이트시킨다. 두 번째는 $\mathbf{s}_{\text{target}}$가 embedding vector 와 가까워지도록 semantic compoenents 를 업데이트시킨다. 업데이트 할 때는 다음 step 에서의  $\mathbf{s}_{\text{target}}$ 가 지금의 $\mathbf{s}_{\text{target}}$ 와 임베딩 벡터의 $\alpha$: $1-\alpha$ 내분점이 되도록, 두 벡터의 difference vector를 각 semantic compoenent vector 에게 균등하게 나눠 더해주어 업데이트 한다. 
semantic component vectors 를 더해서 target semantic vector 를 만드는 모델(summation model) 과 평균으로 만드는 모델(average model) 이 있다. 각 모델의 성능은 Section 5. 에서 확인할 수 있다. 

파라미터와 semantic component update 의 과정은 Figure.\ref{fig:update} 와 Algorithm.\ref{alg:centerloss} 그리고 Algorithm.\ref{func:functions}에서 확인할 수 있다. 

%average model 은 semantic component 가 추가되거나 제거될 때 embedding vector 의 변화가 많이 필요하지 않다는 장점이 있다. 어떤 semantic component 를 갖고 있는지 모르는 Sample 과 semantic component 의 조합으로 쿼링하는 태스크를 수행하기 어렵다는 단점이 있다. 

%\textbf{Statistical view. }

\begin{figure}[!htb] % update
  \includegraphics[width=\columnwidth]{../../figures/update.pdf}
  \caption{update}
  \label{fig:update}
\end{figure}



\begin{algorithm}[!htb]%centerloss
\SetAlgoNoLine
\SetKwFunction{Ftarget}{getTargetSemanticVectors}
\SetKwFunction{Fnew}{getNewSemanticComponentVectors}

\KwIn{Extracted handcrafted features of training data $v_i$ and their semantic components $\mathbf{y}_i$. Initialized parameters $\theta$ in neural embedder. parameters $W, b$ in classifier. initialized semantic component vectors $\mathbf{s}$.Importance coefficients $\textbf{c}$. Hyperparameter $\lambda$, $\alpha$, learning rate $\mu$. }
\KwOut{The parameters $\theta$, $W$, $b$ }
\Repeat{converge}{
    $s_\text{target} = \Ftarget(s_{\mathbf{y}})$ \\
	Compute total loss by $L = L_s + \lambda L_c$\\
	Update parameters $\theta$ by $\theta \leftarrow \theta - \mu \frac{\partial L}{\partial\theta}$ \\
	Update parameters $W$ by $W \leftarrow W - \mu \frac{\partial L}{\partial W}$\\
	Update parameters $b$ by $b \leftarrow b - \mu \frac{\partial L}{\partial b}$\\	
	Update semantic component vectors $s$ by $s \leftarrow \Fnew()$
	

	}
	\caption{The semantics-aware representation vector learning algorithm. }
\label{alg:centerloss}
\end{algorithm}


\begin{algorithm}[!htb]%functions
  \SetAlgoNoLine
  \DontPrintSemicolon
  \SetKwFunction{Ftarget}{getTargetSemanticVectors}
  \SetKwFunction{Fnew}{getNewSemanticComponentVectors}
  
  \SetKwProg{Fn}{Function}{:}{}
  \Fn{\Ftarget{$\mathbf{s}$}}{
    $M \leftarrow $ The number of semantic components \\
  	
    
	\If{Summation Model}
	{$s_{\text{target}} \leftarrow \sum_i^M{s_i}$}
	\ElseIf{Average Model}
	{$s_{\text{target}} \leftarrow \frac{1}{M}\sum_i^M{s_i}$}    
    
    \KwRet $s_{\text{target}}$\;
  }
  \;
    
  \SetKwProg{Pn}{Function}{:}{}
  \Pn{\Fnew{$\mathbf{s}$, $s_{\text{target}}$, $e$, $\alpha$, $\mathbf{c}$}}{
    $M \leftarrow $ The number of semantic components \\
    \ForEach{$i \in \{1, 2, ..., C\}$}{
	  \If{Summation Model}
	  {$s_i \leftarrow s_i - \frac{\alpha}{M}(s_{\text{target}} - e)$}
	  \ElseIf{Average Model}
	  {$s_i \leftarrow s_i - \alpha (s_{\text{target}} - e)$ }
	  \If{Weighted Model}
	  {$s_i \leftarrow \frac{s_i}{||s_i||} c_i$}    
    }
    \KwRet $\mathbf{s}$
  }
  \;  

  \caption{Definitions of functions. }
\label{func:functions}
\end{algorithm}

\section{Evaluation}
이번 섹션에서는 우리가 제안하는 멀웨어 IR 시스템의 성능을 여러 방법으로 평가하여, 앞에서 우리의 시스템의 강점이라고 주장했던 semantic-aware malware retrieval 태스크를 수행하였는지 확인한다. 


\subsection{Implementation and Setup}
\textbf{Datasets. } 
모델의 학습과 평가에 사용된 dataset은 VirusTotal로부터 [언제부터 언제까지] crawling된 샘플들 중 security expert가 검수하여 확진을 내린 샘플들로 구성되어있다. 또한 VirusTotal에서 제공하는 수십개 이상의 AntiVirus 엔진의 탐지결과로부터 Labeling을 자동화 하였다. VirusTotal로부터 각 AntiVirus 엔진들이 탐지해낸 결과를 의미있는 word 단위로 파싱하고, 샘플들로 부터 얻어진 word token들을 대상으로 security expert가 의미있는 토큰과 그렇지 않은 토큰을 선별하는 작업을 수행하였다. 추가로, 의미 있는 토큰에 대해 중요도에 따라 우선순위를 매기도록 하였다. 이를 통해 학습 샘플에 대해 expert가 일일히 수동으로 label을 다는 시간을 줄이면서도 높은 accuracy의 label을 얻을 수 있다. 또한, 하나의 샘플에 대표 레이블과 멀티레이블 두 종류의 레이블을 부여했으며, 멀티 레이블에는 대표 레이블이 포함된다. 

\begin{itemize}
	\item{ \textbf{Dataset 1. PE1300. } VirusTotal로부터 crawling된 샘플들 PE 5만여개 중 security expert가 검수하여 확진을 내린 PE 샘플 1300여개 11종의 labels. 이 중 8개는 대표 레이블이며 대표 레이블에 대한 샘플들의 수는 거의 같다. 즉 대표 레이블에 대한 샘플의 분포는 유니폼하다.
	}
	\item{ \textbf{Dataset 2. APK19000. } VirusTotal로부터 crawling된 샘플들 APK 8만여개 중 security expert가 검수하여 확진을 내린 APK 샘플 2만여개 83종의 labels. 이 중 19개는 대표 레이블이며 대표 레이블에 대한 샘플의 수는 1000개로 같다. PE 데이터셋과 마찬가지로 대표 레이블에 대한 샘플의 분포는 유니폼하다.  
	}
\end{itemize}



\textbf{Hyperparameters. }
- layers, learning rate, train/valid ratio, Label importances, 


\textbf{Hancrafted Feature Extraction}
- Thumbnail
- Size, Entropy
- Call histogram


\subsection{Auxiliary Task Evaluation}
Auxiliary task 의 정확도는 더 안좋지만 쿼리 결과는 더 좋은 경우가 있다. 이는 Malware Semantic space의 구축 상태를 평가하기에 Auxiliary Task 의 정량 평가만으로는 부족하다는 것을 의미한다. 


\subsection{Querying Quantitative Test}
- metric performance measures
	- acc, precision, recall, auc, 등 label 겹치는 정도를 수치로 정량화
	- inner class variance 측정 후 센터로스 유무에 대해 비교. 
- Single label, Multi label, Centerloss(Mean, Add) , Weighted Center loss(Mean, Add)
- Train data querying , Valid data querying

앞서 Malware Semantic Space 란 사람이 중요하다고 생각하는 레이블이 많이 겹칠 수록 가까이 있도록, 그리고 그렇지 않을 수록 멀리 있도록 멀웨어 샘플들의 표현 벡터를 위치시킨 공간이라고 설명한 바 있다. 그리고 Malware Semantic Space 의 표현 벡터들을 이용하여 MIR 을 구축했을 때 중요한 레이블이 비슷한 샘플들을 검색할 수 있을 것이라고 했다. MWC Loss 를 사용하여 MIR System 을 구축했을 때 위의 정량 수치가 대체적으로 더 높다는 것은 곧 다른 Objective function 을 사용했을 때보다 Semantics 을 더 잘 학습했다고 볼 수 있고 이는 MIR System 의 Desired property 1 번인 Semantic Understanding 속성을 다른 로스들에 비해 더 만족시킨다.  


\subsection{Querying Qualitative Test}
\textbf{Visualizations.}
- PCA : Variance of inner class 가 작고 variance of inter class 가 크다는 것을 눈으로 보여줌. 더불어 주요 태그를 함께 보유하고 있다면 보틀넥이 가까운 곳에 위치함을 보여줌으로써 우리의 제안 방법을 통해 Malware Semantic space 가 구축되었다는 것을 보여줌.
vector space 에서 연산 관계가 있음을 보여줌.  

\textbf{Querying by malware sample. }
MWC 를 사용하여 학습시킨 MIR 시스템과 그렇지 않은 시스템에 대하여 멀웨어 샘플을 쿼링하여 얻은 top k 개의 결과를 리스팅하여 기재함으로써 MWC Loss 가 Malware Semantic space 를 구축하는데 도움이 된다는 것을 보여준다.


\textbf{Querying by labels. }
레이블들을 쿼링하여 동일 레이블 조합을 갖고 있는 top k 개의 결과를 리스팅하여 기재함으로써 단순히 Data retrieval system이 아니라 Information retrieval system 을 설계한 것이라는 것을 보여줌. 


\subsection{Center Vector Combination Methods}
Mean, Add 방법의 차이를 위의 결과들로부터 설명.

\textbf{add center vectors. } 



\subsection{Generalization}
입력 Feature 가 Semantics-aware feature 일 수록 새로운 샘플에 대한 Error 가 더 작다. 우리가 제안한 목적함수는 멀티 레이블과 레이블 별 중요도로부터 해당 샘플이 시멘틱 스페이스에서 어디에 위치해야하는지를 가이드해준다. 따라서 Train Samples 에 대해서는 데이터의 입력 Feature가 Semantic-aware 하지 않더라도 원하는 곳에 특징 표현 벡터를 위치시킬 수 있다. 하지만 새로 보는 샘플의 representation vector 가 우리가 원하는 곳에 위치하게 된다는 보장은 없다. 딥러닝의 하이러키컬 특징 표현 학습이 MWC loss 를 통해 가이드 되는 Semantic을 학습할 수 있도록 하려면 입력 Feature가 충분히 Semantics-aware 해야 하고, 이에 관련된 연구들은 섹션2에서 소개하였다. 

우리는 이 차이를 보이기 위해 semantics-aware level 이 다른 두 가지 피쳐에 대해 벡터 표현을 학습시키고, validation set 의 벡터 표현이 Semantic space 위에 잘 위치하는지 확인하는 실험을 한다. 첫 번째 실험에 사용되는 피쳐는 간단한 정적 분석을 통해 얻을 수 있는 Size, entropy\citep{} 이고, 두 번째 실험에 사용되는 피쳐는보다 조금 더 Semantics-aware 한 피쳐인 Thumbnail\citep{} 이다. 위에서 진행한 정량 평가와 정성 평가를 두 실험에 대해 진행해 본 결과 더 semantics-aware한 피쳐를 입력으로 사용할 수록 평가의 결과가 좋았고 이는 즉 Generalization 이 더 된다는 것을 의미한다. 즉, 누구라도 더 Semantics-aware한 피쳐를 입력으로 넣어줄 수 있다면, 간단히 Add-on 가능한 MWC loss 를 사용하여 훌륭한 Semantic space 를 구축할 수 있게 된다. 

% 시간이 된다면 APK Call-graph 도... 하면 좋을텐데..  


\section{Future Works}

The current method requires that all labels be correct. If not, the model can be vulnerable to noise. But getting the correct label in malware domain is not easy. Therefore, label noise-robust model should be created to estimate the real label of malware samples from features of malware through denoising. If we create MR system through this, we could make more semantic-aware malware information system. In addition, quick response to new samples is an important part of the MR system, but in case of deep learning model, it takes time to train. It is also our future work to solve by using continual learning or online update.
\section{Related Works}
Nataraj et al. 은 large scale malware search and retrieval system을 제안하였다. 이 논문에서는 malware image로 부터 fingerprints를 얻고, 이를 nearest neighbor search를 통해 비슷한 샘플을 retrieval하는 방법을 제안하였다. Upchurch et al. 은 similarity testing을 통해 variant malware를 탐지하는 framework를 소개했다. 이 framework는 BitShred, TLSH, sdhash, ssdeep 등의 방법으로 정적 feature를 추출하고 이를 비교함으로써 유사한 malware인지의 여부를 판단하였다. Palahan et al. 은 system call dependency graph로부터 significant malicious behaviors를 추출하고 이를 비교함으로써 malware간의 similarity를 비교하는 방법을 제시했다. 
Neural IR
Multilabel embedding



\section{Conclusions}

우리는 이 페이퍼에서, deep learning 을 활용하여 malware information retrieval system 을 만드는 방법을 소개했다. 그리고 시스템이 semantic understanding 속성을 갖게 하기 위하여, 제안된 Loss function 인 multilabel centerloss 를 사용하여 semantic space 를 approximate 하였다. 우리가 제안한 방법으로 만든 MIR 시스템에 멀웨어 샘플, 시멘틱 컴포넌트들, 멀웨어 샘플과 시멘틱 컴포넌트의 조합 이렇게 세 가지 방법으로 쿼리를 하였다. 그리고 정량, 정성 평가 결과를 통해 제안된 방법이 시스템을 semantic-aware 하게 만들었음을 보였다. 

\appendix

\section{Location}

Note that in the new ACM style, the Appendices come before the References.


\begin{acks}
% TODO: For the submission, don't include acknowledgments since they would most likely deanonymize you.
\end{acks}
 % TODO: replace with your brilliant paper!

\bibliographystyle{ACM-Reference-Format}
\bibliography{ccs-sample}

\end{document}
