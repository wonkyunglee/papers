\section{Problem Settings}

우리가 제안하는 malware IR 시스템은 벡터 학습 과 리트리벌 두 페이즈로 나뉜다. 두 페이즈에서 사용될 노테이션들을 정의하고, 이들을 이용해서 각 페이즈에서 어떤 태스크를 행해야하는지를 설명한다. 

\subsection{Notations}

우리는 먼저 학습 멀웨어 셋으로 Y를 갖고 있다. 그리고 각 멀웨어에 해당하는 싱글레이블 셋  Ys 과 멀티레이블 셋 Ym 이 있다. 이 레이블을 얻는 방법은 섹션 6.1에서 자세히 설명한다. 그리고 각 멀웨어로부터 손으로 추출된 피쳐의 튜플들의 셋을 V 로 정의한다. 어떤 피쳐들을 추출해서 사용했는지는 섹션 6.2에서 설명한다. 

멀웨어 IR 시스템은  4개의 원소를 갖는 tuple 로 정의된다. h 는 멀웨어 피쳐로부터 벡터 representation 을 얻을 수 있는 임베더 함수이다. E 는 검색이 될 멀웨어의 임베딩 벡터 셋이다. d 는 입력된 쿼리와 임베딩 벡터 간 거리를 계산하는 함수이다. R 은 리트리벌 결과를 반환하는 프레임워크이다. 학습 페이즈에서는 적절한 h 를 학습하고 이를 통해 Z로부터 E를 얻는다. 리트리벌 페이즈에서는 멀웨어 샘플 쿼리 q 를 입력받고 가장 가까운 k 개의 neighbor 를 랭킹 모듈 R 을 통해 반환한다.


\subsection{Tasks}

학습페이즈에서는 IR system 에서 사용하기 위한 벡터 리프레젠테이션을 얻기 위한 Auxiliary Task를 수행한다. Auxiliary Task 는 멀웨어 분류 태스크를 학습하는 것으로 싱글 레이블 클래시피케이션이나 멀티레이블 클래시피케이션이 될 수 있다. f 인 f 는 멀웨어 피쳐로부터 레이블을 추정하는 함수이다. 함수 f 는 다시 g 와 h 의 합성 함수로 정의할 수 있다. h 는 위에서 설명한, 멀웨어 피쳐로부터 vector representation 을 얻을 수 있는 임베딩 함수이다. g는 임베딩으로부터 label 을 추정할 수 있는 클래시파이어이다.  
f(Z)

리트리벌 페이즈에서는 멀웨어 샘플 쿼리 q를 입력받아 학습 페이즈에서 구했던 h 함수를 이용하여 임베딩한다. 임베딩된 쿼리 eq 와 E 의 원소들 간 거리를 d 함수를 통해 측정하고, 가까운 k 개의 neighbor 를 랭킹 모듈 R 을 통해 반환한다. 
Result tuples = (Xvecj ) where


