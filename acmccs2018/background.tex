

\section{Background}


\subsection{Deep learning for Classification }


\subsection{IR System using Vector space model}
Vector representation 을 사용하지 않는 IR 시스템(ex. Standard boolean model)은 다음과 같은 단점이 있었다.
\begin{itemize}
\item Exact matching may retrieve too few or too many documents
\item hard to translate a query into a boolean expression
\item all terms are equally weighted
\item more like data retrieval than information retrieval
\end{itemize}


이를 극복하기 위해 [Vector space model] 에서는 Vector representation[ref]을 사용하여 IR 시스템을 만들었으며 이는 다음과 같은 장점들이 있다.

\begin{itemize}
\item Simple model based on linear algebra
\item Term weights not binary
\item Allows computing a continuous degree of similarity between queries and documents
\item Allows ranking documents according to their possible relevance
\item Allows partial matching
\end{itemize}


앞서 설명했듯, 딥러닝을 사용하여 머신러닝 태스크를 수행하면 hierarchical representation 을 배울 수 있기 때문에 Generalization 이 더 잘 되는 장점이 있다. 따라서 Vector space model 을 이용하는 여러 태스크들과 딥러닝은 조합이 좋았고, 이로인해 많은 태스크에서 성능 상 breakthrough 를 이루어냈다.[refs : imagenet, word2vec, speech recognition, ...]. 더불어 Vector space model 을 사용하는 IR 분야 또한 딥러닝과 만나면서 한번 더 Breakthrough를 이루어냈다.[refs : document retrieval, Retrieval based QA, Image retrieval]. 

벡터 representation 을 학습시키는 방법으로는 supervised 와 unsupervised 가 있다. Unsupervised 로 가르치는 대표적인 방법은 Autoencoder를 이용하는 것으로, 이를 활용해서 멀웨어를 인코딩하려는 시도[ref]가 있었다. Supervised 로 가르치는 방법은 인간이 레이블을 주고 그 레이블로의 Classification 혹은 Regression 태스크를 수행하는 딥러닝 모델을 만들어, 정보를 가장 압축적으로 담고있는 bottleneck layer 의 아웃풋을 해당 샘플의 representation vector 로 보는 것이다. 이 방식은 vector representation 의 위치를 가르치는 사람이 어느정도 컨트롤 할 수 있게 된다. 예컨데 메트릭 러닝을 위한 objective function 을 사용해서 어떤 샘플들을 가깝게 보내고 어떤 샘플들을 멀게 보낼지를 결정할 수 있고[siamese, triplet, centerloss ...], 이는 one shot or few-shot learning 의 태스크를 수행할 수 있기 때문에, 벡터 간 distance 를 기준으로 랭킹을 결정하고 적은 샘플에도 대응할 수 있어야 하는 information retrieval 의 성능을 향상시킬 수 있다. 


\subsection{Metric Learning}


