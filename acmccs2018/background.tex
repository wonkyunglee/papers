\section{Background}

% Handcrafted Feature Extract
\subsection{Deep learning }

 neural networks는 훈련 데이터의 통계적 특성을 기반으로 특징을 추출할 수 있는 machine learning 방법이다. 또한 DNN은 network를 수 층의 hidden layers로 구성함으로써 hierarchical feature representation을 얻을 수 있다. 일반적으로 DNN은 input layer와 수 개의 hidden layer, output layer로 구성된다. Classification task의 경우, input layer는 task의 목적이 되는 sample의 feature vector representation을 입력으로 받는다. Output layer는 주어진 input vector에 관계된 class의 probability를 vector로 출력한다. 이를 통해 DNN은 input vector에 따른 class를 predict 할 수 있다.
 
특히 Image classification task의 경우 Convolutional network를 사용함으로써 모델의 성능을 향상시킬 수 있다.[ref?] Convolutional network는 convolutional layer와 pooling layer, 일반적인 neural network layer로 구성되며, convolutional layer의 경우 filter를 input에 대해 convoluting함으로써 feature를 추출한다. Pooling layer의 경우 영향을 적게 주는 feature를 down sampling하는 효과를 얻는다.


\subsection{Information Retrieval}
Information retrieval 은 본래 유저의 string  쿼리로부터 관련 문서를 찾는 태스크로 자연어 처리 분야에서 발전하였다. 하지만 이를 다른 도메인으로 확장시켜 이미지\cite{datta2008image, yu2015learning}, 음악\cite{schedl2014music}, 의료\cite{goeuriot2016medical, mourao2015multimodal}, 멀웨어\cite{santos2013noa} 등 다양한 분야에서 사용되고 있다. 

\textbf{Boolean Model. }Information Retrieval System 을 만들기위한 모델로, Boolean model 이 제안되었다. boolean model 은 boolean 로직과 집합 이론을 배경으로 하여 문서들과 유저의 쿼리들을 집합으로 대하고 검색을 시행하는 모델이다. 이는 직관적이고 구현하기 쉽다는 장점이 있었지만, 유사도가 Discrete 하게 계산되어 리트리벌 결과가 아주 많아지거나 정확히 일치하지 않으면 검색되지 않는 현상이 일어나고, 모든 term 들이 똑같은 비중을 가지는 한계들이 있었다.\cite{lashkari2009boolean, } 


\textbf{Vector Space Model.} 또 다른 방법으로 Vector space model 이 있다. 이는 문서들과 유저의 쿼리를 모두 벡터스페이스에서 표현함으로써 벡터 간 거리가 곧 유사도가 되도록 하는 모델이다. 선형 대수에 기반한 간단한 모델로, 텀에 대한 가중치를 다르게 줌으로써 리트리벌 결과를 더 좋게 만들 수 있고, 쿼리의 유사도를 continuous Value 로 나타낼 수 있다는 장점이 있다.\cite{salton1975vector} 문서와 유저의 쿼리를 벡터로 만드는 방법\cite{}과 벡터 간 거리를 측정하는 방법\cite{}에 따라 여러가지 모델들이 존재한다.


%\texttt{
%\begin{itemize}
%\item Exact matching may retrieve too few or too many documents
%\item hard to translate a query into a boolean expression
%\item all terms are equally weighted
%\item more like data retrieval than information retrieval
%\end{itemize}}



%\begin{itemize}
%\item Simple model based on linear algebra
%\item Term weights not binary
%\item Allows computing a continuous degree of similarity between queries and documents
%\item Allows ranking documents according to their possible relevance
%\item Allows partial matching
%\end{itemize}


\textbf{Learn Vector Representation Using Deep Learning.} 앞서 설명했듯, 딥러닝을 사용하여 머신러닝 태스크를 수행하면 hierarchical representation 을 배울 수 있기 때문에 Generalization 이 더 잘 되는 장점이 있다. 따라서 Vector space model 을 이용하는 여러 태스크들과 딥러닝은 조합이 좋았고, 이로인해 많은 태스크에서 성능 상 breakthrough 를 이루어냈다.[refs : imagenet, word2vec, speech recognition, ...]. 더불어 Vector space model 을 사용하는 IR 분야 또한 딥러닝과 만나면서 한번 더 Breakthrough를 이루어냈다.[refs : document retrieval, Retrieval based QA, Image retrieval]. 

벡터 representation 을 학습시키는 방법으로는 supervised 와 unsupervised 가 있다. Unsupervised 로 가르치는 대표적인 방법은 Autoencoder를 이용하는 것으로, 이를 활용해서 멀웨어를 인코딩하려는 시도[ref]가 있었다. Supervised 로 가르치는 방법은 인간이 레이블을 주고 그 레이블로의 Classification 혹은 Regression 태스크를 수행하는 딥러닝 모델을 만들어, 정보를 가장 압축적으로 담고있는 bottleneck layer 의 아웃풋을 해당 샘플의 representation vector 로 보는 것이다. 이 방식은 vector representation 의 위치를 가르치는 사람이 어느정도 컨트롤 할 수 있게 된다. 예컨데 메트릭 러닝을 위한 objective function 을 사용해서 어떤 샘플들을 가깝게 보내고 어떤 샘플들을 멀게 보낼지를 결정할 수 있고[siamese, triplet, centerloss ...], 이는 one shot or few-shot learning 의 태스크를 수행할 수 있기 때문에, 벡터 간 distance 를 기준으로 랭킹을 결정하고 적은 샘플에도 대응할 수 있어야 하는 information retrieval 의 성능을 향상시킬 수 있다. 


\subsection{Metric Learning}


