
\section{Good Properties for Malware IR systems}
멀웨어 IR 시스템은 한계들이 존재했다. 특히 Raw binary file 로부터 좋은 피쳐를 뽑기가 수월하지 않고, intraclass variance 는 크고 innerclass variance 는 작으며 계속해서 변종과 새로운 종이 생겨나는 멀웨어 도메인의 특징으로 말미암아 좋은 멀웨어 IR 시스템이 가져야 할 속성들은 다른 도메인의 IR 시스템과는 조금 다르다.

\textbf{Semantic understanding. }
쿼링 샘플에 대해 구조적으로 비슷한 샘플 뿐만 아니라 의미적으로도 비슷한 샘플들을 랭킹하고 retrieve 할 수 있어야 한다. 여기서 의미가 비슷하는 것은 멀웨어의 행동 혹은 사람이 생각하기에 중요한 멀웨어의 속성들이 비슷하다는 것을 의미한다. 


\textbf{Robustness to novelty. }
새로운 변종들은 계속해서 나타나기 때문에 이에 빠르게 대응할 수 있어야 한다. 


\textbf{Robustness to rare inputs. }
멀웨어 도메인에서의 샘플 수는 그 멀웨어 패밀리의 영향력을 의미하지 않는다. 물론 비례하는 경향이 없는것은 아니지만, 하나의 멀웨어가 전세계적으로 유행할 수도 있고, 변종은 많지만 별로 영향력이 없는 멀웨어일 수도 있다. 따라서 해당 패밀리의 샘플 수가 적다고해서 쿼링 결과에서 랭킹 순위가 밀려서는 안된다. 즉 적은 수의 샘플에 대해서도 모델은 강건해야한다. 


\textbf{Robustness to polymorphism. }


\textbf{Robustness to variable size inputs. }
멀웨어의 raw file size Variance 는 매우 크다. 작게는 몇키로바이트부터 크게는 기가단위까지 갈 수 있다. 사이즈에 상관없이 retrieve 를 할 수 있어야 한다.

\textbf{Efficiency. }
세상에 존재하는 멀웨어 샘플 개수는 너무나도 많고 기하급수적으로 늘어나고 있기도 하다. 따라서 수많은 샘플들에서 k 개의 상위 랭크된 결과를 적절한 시간 내에 retrieve 하려면 랭킹 모듈이 효율적이어야한다.
